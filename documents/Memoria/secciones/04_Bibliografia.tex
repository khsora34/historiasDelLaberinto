\begin{thebibliography}{99}
\bibitem{P} Publicaciones utilizadas en el estudio y desarrollo del trabajo.
Hay que utilizar un sistema internacional para referencias bibliográficas, de acuerdo con las indicaciones del tutor. Por ejemplo, el \textbf{sistema de IEEE}.

\bibitem{GR} M. de Guzmán y B. Rubio, \textit{Integración: Teoría y Técnicas}, Alhambra, Madrid, 1979.

\bibitem{Ma} P. Mattila, \textit{Geometry of Sets and Measures in Euclidean Spaces}, Cambridge University Press, Cambridge, 1995.

\bibitem{Ro} C.A. Rogers, \textit{Hausdorff Measures}, Cambridge University Press, Cambridge, 1998.

\bibitem{unrealEngineHomepage} Epic Games Inc. (2004). Unreal Engine [Online]. Available: https://www.unrealengine.com

\bibitem{unity3dHomepage} Unity Technologies (2005, junio, 8). Unity 3D [Online]. Available: https://unity.com

\bibitem{Alberto_Carrasco} Alberto Carrasco Carrasco, \textit{https://blogs.upm.es/observatoriogate/2018/07/04/que-es-un-motor-de-videojuegos/}, Universidad Politécnica de Madrid, Madrid, 2018

\bibitem{mediaPlayer} Apple Inc., \textit{https://developer.apple.com/documentation/mediaplayer/}, Cupertino, California, Estados Unidos
\bibitem{npcGeekno} SUMMON PUBLISHING GROUP, \textit{https://www.geekno.com/glosario/npc}, Valencia, 2019

\bibitem{nonograms} James Dalgety, \textit{https://www.puzzlemuseum.com/griddler/gridhist.htm}, 2017

\bibitem{combinationPuzzles} Laura Martín Hoz, \textit{http://museodeljuego.org/wp-content/uploads/ROMPECABEZAS-DE-COMBINACIÓN-Laura-mart\%C3\%ADn.docx}, 2018

\end{thebibliography}
